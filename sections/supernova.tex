\section{Süpernova Patlamaları}

Süpernova, yıldızların yaşamlarının son evrelerinde meydana gelen, çok büyük miktarda enerji açığa çıkaran kozmik patlamalardır. Bu patlamalar bir yıldızın tüm ışınım gücünü bir anda milyonlarca katına çıkararak galaksiler arası ölçekte gözlenebilir hale getirir. Süpernovalar hem ağır elementlerin sentezlenmesinde hem de bu elementlerin galaktik ortama yayılmasında temel rol oynar.

Süpernovalar genel olarak iki ana kategori altında incelenir: \textbf{Tip I} ve \textbf{Tip II}. Bu sınıflandırmanın temel kriteri patlama sırasında tayf çizgilerinde \textbf{hidrojen (H)} çizgilerinin bulunup bulunmamasıdır.

\subsection{Tip I Süpernova}
Tip I süpernovaların tayfında \textbf{hidrojen çizgileri yoktur}. Genellikle bir beyaz cücenin kritik kütle sınırına (Chandrasekhar limitine) ulaşarak kararsız hale gelmesi ve termonükleer bir patlamaya dönüşmesi sonucu gerçekleşir.

\subsubsection{Tip Ia}
Tip Ia süpernovalar, bir \textbf{beyaz cücenin} ikili yıldız sisteminde, yoldaş yıldızdan madde çekerek kütlesinin $1.4 M_{\odot}$ seviyesine ulaşmasıyla gerçekleşir. Bu durumda:
\[
M > M_{\text{Chandrasekhar}} \Rightarrow \text{Termonükleer Patlama}
\]
Bu patlama karbon ve oksijen füzyonunun kontrolden çıkmasıyla tüm yıldızın yok olmasına neden olur. Tip Ia süpernovalar standart mum (standard candle) olarak kozmolojik uzaklık ölçümlerinde kullanılır.

\subsubsection{Tip Ib}
Tip Ib süpernovaların tayfında hidrojen çizgisi bulunmaz ancak \textbf{helyum (He)} çizgileri vardır. Bu tip patlamalar, kütle kaybı yaşamış ya da yıldız rüzgarları tarafından dış zarfları soyulmuş \textbf{büyük kütleli yıldızların} çekirdek çökmesi sonucu oluşur.

\subsubsection{Tip Ic}
Tip Ic süpernovaların tayfında ne hidrojen ne de helyum çizgileri bulunur. Bu sınıf, yıldızın hem hidrojen hem de helyum zarflarını kaybetmiş olduğu çok daha şiddetli soyulmuş kütleli yıldızların çekirdek çökmesiyle ilişkilidir. Genellikle Wolf–Rayet yıldızları ile ilişkilendirilir.

\subsection{Tip II Süpernova}
Tip II süpernovaların tayfında \textbf{hidrojen çizgileri belirgindir}. Bu patlamalar, büyük kütleli bir yıldız ($M \gtrsim 8M_{\odot}$) yakıtının tükenmesiyle çekirdeğinin çökmeye uğraması sonucu gerçekleşir. Çekirdekte füzyon sona erdiğinde:
\[
P_{\text{iç}} \not\geq P_{\text{gravitasyon}} \Rightarrow \text{Çekirdek Çökmesi}
\]

Çökme sırasında elektronlar protonlarla birleşerek nötronlara dönüşür:
\[
e^- + p^+ \rightarrow n + \nu_e
\]

Sonuçta:
\begin{itemize}
    \item Çekirdekte \textbf{nötron degenerasyon basıncı} oluşur.
    \item Dış katmanlar şok dalgasıyla uzaya saçılır.
    \item Geriye \textbf{nötron yıldızı} veya çok büyük kütlede ise \textbf{kara delik} kalır.
\end{itemize}

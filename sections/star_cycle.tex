\section{Yıldız Döngüsü}

\begin{itemize}
    \item \textbf{Doğum (Prototip Evre):} Nebulanın çökmesiyle oluşur.
    
    \item \textbf{Protostar:} Füzyon henüz başlamamıştır. Çekimsel ısınma olur.
    
    \item \textbf{Ana Kol:} Çekirdek sıcaklığı yaklaşık olarak
    \[
        T \approx 10^7\,\mathrm{K}
    \]
    olur. Hidrojen füzyonu başlar:
    \[
        \mathrm{H} \rightarrow \mathrm{He}
    \]
    Yıldız dengededir:
    \[
        P_{\textrm{iç}} = F_{\textrm{kütleçekimi}}
    \]
    Yıldız yaşamının yaklaşık \%90'ı bu evrede geçer. Hidrojen tükendiğinde helyum birikir, çekirdek çöker ve dış katmanlar genişler. Yıldız kırmızı dev evresine geçer.
    
    \item \textbf{Kırmızı Dev:} Helyum füzyonu başlar:
    \[
        3\mathrm{He} \rightarrow \mathrm{C}
    \]
    Dış katmanlar genişlemiş ve soğumuştur. Yıldızın boyutu ve parlaklığı artar. Helyum tükendiğinde çekirdekte C ve O oluşur. Füzyon sona erer. Dış katmanlar uzaya atılır (Gezegenimsi Bulutsu). Geriye sadece çekirdek kalır (Beyaz Cüce).
\end{itemize}

\[
\textrm{Nebula} \;\rightarrow\; \textrm{Protostar} \;\rightarrow\; \textrm{Ana Kol} \;\rightarrow\; \textrm{Kırmızı Dev} \;\rightarrow\; \textrm{Beyaz Cüce}
\]

\documentclass[a4paper,12pt]{article}

% Burada paketler var 

\begin{document}

\begin{titlepage}
    \centering
    
    {\Large \textbf{Astrofizik Üzerine El Yazmalarım}}\\[2.5cm]
    
    {\large Kasım 2025}\\[2cm]
    
    {\large \textbf{Yazar:} Emre SEZER}\\[8cm]
    
    \vfill
\end{titlepage}

\section{Yıldız}

 Ağırlıklı olarak H ve He'dan oluşan, ışık ve ısı yayan, kararlı plazmalardır. Çekirdeklerinde füzyon meydana gelir. Açığa çıkan
enerji yıldızın yüzeyine ulaşır ve dış uzaya radyasyon ile yayılır. Yıldızın tayfı(spektrumu) ve parlaklığı ile kütlesi, yaşı, kimyasal bileşeni vb. özellikleri belirlenebilir. Birim olarak güneş kütlesi, güneş yarıçapı ve güneş parlaklığı kullanılır.

\section{Yıldız Döngüsü}

\begin{itemize}
    \item \textbf{Doğum (Prototip Evre):} Nebulanın çökmesiyle oluşur.
    
    \item \textbf{Protostar:} Füzyon henüz başlamamıştır. Çekimsel ısınma olur.
    
    \item \textbf{Ana Kol:} Çekirdek sıcaklığı yaklaşık olarak
    \[
        T \approx 10^7\,\text{K}
    \]
    olur. Hidrojen füzyonu başlar:
    \[
        \text{H} \rightarrow \text{He}
    \]
    Yıldız dengededir:
    \[
        P_{\text{iç}} = F_{\text{kütleçekimi}}
    \]
    Yıldız yaşamının yaklaşık \%90'ı bu evrede geçer. Hidrojen tükendiğinde helyum birikir, çekirdek çöker ve dış katmanlar genişler. Yıldız kırmızı dev evresine geçer.
    
    \item \textbf{Kırmızı Dev:} Helyum füzyonu başlar:
    \[
        3\text{He} \rightarrow \text{C}
    \]
    Dış katmanlar genişlemiş ve soğumuştur. Yıldızın boyutu ve parlaklığı artar. Helyum tükendiğinde çekirdekte C ve O oluşur. Füzyon sona erer. Dış katmanlar uzaya atılır (Gezegenimsi Bulutsu). Geriye sadece çekirdek kalır (Beyaz Cüce).
    
\end{itemize}

\[
\text{Nebula} \;\rightarrow\; \text{Protostar} \;\rightarrow\; \text{Ana Kol} \;\rightarrow\; \text{Kırmızı Dev} \;\rightarrow\; \text{Beyaz Cüce}
\]





\end{document}
